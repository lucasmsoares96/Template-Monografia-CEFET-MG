\begin{abstract}
    O resumo deve ressaltar o objetivo, o método, os resultados e as conclusões do documento. A ordem e a extensão destes itens dependem do tipo de resumo (informativo ou indicativo) e do tratamento que cada item recebe no documento original. Deve ser precedido da referência do documento, com exceção do resumo inserido no próprio documento, e ser composto de uma sequência de frases concisas, de cunho afirmativo e sem enumeração de tópicos, dado que se recomenda o uso de parágrafo único. As palavras-chave devem figurar logo abaixo do resumo, antecedidas da expressão palavras-chave, e finalizadas também por ponto.
    É importante evitar:
    \begin{enumerate}
        \item símbolos e contrações que não sejam de uso corrente;
        \item fórmulas, equações, diagramas e similares que não sejam absolutamente necessários; quando seu emprego for  imprescindível, deve-se defini-los na primeira vez em que aparecerem.
    \end{enumerate}
    Quanto à extensão, os resumos devem ter:
    \begin{enumerate}
        \item de 150 a 500 palavras os de trabalhos acadêmicos (teses, dissertações e outros) e relatórios técnico-científicos;
        \item de 100 a 250 palavras os de artigos de periódicos;
        \item de 50 a 100 palavras os destinados a indicações breves.
    \end{enumerate}
    Como tratado, o resumo deve ser seguido das palavras representativas do conteúdo do trabalho, isto é, palavras-chave, ou descritores, no idioma em que foi redigido (mínimo 3). Elas devem ser separadas por ponto e virgula e finalizadas com ponto final.
    \\ \\
    \textbf{Palavras-chave:} Palavra-chave 1; Palavra-chave 2; Palavra-chave 3; Palavra-chave 4; Palavra-chave 5.
\end{abstract}